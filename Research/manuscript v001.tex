\documentclass[a4paper,11pt]{article}

 
\begin{document}
\title{Considering Computational Intelligence}
\author{Jessica E. Snyder}
\date{December 31, 2020}
\maketitle

\begin{abstract}
The objective of this exercise is to learn about computation design. For an hour a day for 100 days, the author plans to research and document this subject. If after 100 days the author has nothing to add to this topic, that is interesting. Else: the author will have a stockpile of new thoughts and understanding for the next 100 days.
\end{abstract}

Keywords: Computational intelligence, computational design

\tableofcontents

\section{Let's begin}
Computational processes extend the patterns human see across larger and larger sets of numeric values. What if humans could ask the computational processes for help finding the patterns, instead of giving instructions? Maybe we already have.  


\section{How do bees make decisions?}

The natural order has evolved solutions decision making skills - like way-finding for flying insects. Understanding the balancing act between numerous data streams (i.e. threat level from predators, environmental hazards, maintaining a mental map of the area, acheiving a goal, finding food) provides a starting point to program robotic swarms for cooperation. 

"Computational intelligence techniques have widespread applications in the field of engineering process optimization, which typically comprises of multiple conflicting objectives. An efficient hybrid algorithm for solving multi‐objective optimization, based on particle swarm optimization (PSO) and artificial bee colony optimization (ABCO) has been proposed in this paper. The novelty of this algorithm lies in allocating random initial solutions to the scout bees in the ABCO phase which are subsequently optimized in the PSO phase with respect to the velocity vector. The last phase involves loyalty decision‐making for the uncommitted bees based on the waggle dance phase of ABCO. This procedure continues for multiple generations yielding optimum results. The algorithm is applied to a real life problem of intercity route optimization comprising of conflicting objectives like minimization of travel cost, maximization of the number of tourist spots visited and minimization of the deviation from desired tour duration. Solutions have been obtained using both pareto optimality and the classical weighted sum technique. The proposed algorithm, when compared analytically and graphically with the existing ABCO algorithm, has displayed consistently better performance for fitness values as well as for standard benchmark functions and performance metrics for convergence and coverage."\cite{beed2020hybrid}



\bibliographystyle{acm}
\bibliography{references}

\newpage

\end{document}